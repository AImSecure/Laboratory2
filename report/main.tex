% main.tex

% Use the ACM large 1-column format class from your template directory
\documentclass[acmlarge]{template/column-format-template/acmart}

% Set up any additional package imports here
\usepackage{graphicx}     % For including graphics
\usepackage{amsmath}      % For advanced math formatting
\usepackage{algorithm}    % For algorithm pseudocode
\usepackage{algorithmic}  % For algorithm pseudocode
\usepackage{booktabs}     % For professional tables
\usepackage{listings}     % For code listings
\usepackage{subcaption}   % For subfigures
\usepackage{url}          % For URLs in bibliography
\usepackage{titlesec}     % For custom formatting
\usepackage{xcolor}       % For colors
\usepackage{listings}     % For code snippets
\usepackage{subcaption}   % For subfigures and subtables
\usepackage{hyperref}     % For links and references
\usepackage{enumitem}     % For customizing list formatting and styles

% Custom settings for listings
\lstset{
    language=Python,                    % Sets the programming language for syntax highlighting
    basicstyle=\footnotesize\ttfamily,  % Sets the font style and size for the code
    keywordstyle=\color{blue},          % Style applied to keywords (e.g., "def", "import")
    commentstyle=\color{gray},          % Style applied to comments (e.g., lines starting with "#")
    stringstyle=\color{red},            % Style applied to strings (e.g., text within quotes)
    % backgroundcolor=\color{gray!10},  % Background color for the code block (light gray in this case)
    % numbers=left,                     % Displays line numbers on the left side of the code block
    % numberstyle=\tiny,                % Font size/style for line numbers
    % stepnumber=1,                     % Line number increment (e.g., every 1 line gets a number)
    % numbersep=5pt,                    % Distance between the line numbers and the code
    linewidth=0.95\linewidth,           % Code block occupies 80% of text width
    xleftmargin=-20pt,                  % Moves the listing 20pt closer to the left margin
    xrightmargin=5pt,                   % Optional: Adds extra padding on the right
    breaklines=true,                    % Automatically breaks long lines to fit within the page width
    captionpos=b,                       % Places the caption below the code block
    abovecaptionskip=5pt,               % Adjusts the space above the caption to 5pt
    belowcaptionskip=8pt,               % Adjusts the space below the caption to 8pt
}

% Custom settings for hyperref
% FIXME: update and fix
\hypersetup{
    colorlinks=false,            % Disable colored links (use borders instead)
    linkbordercolor=1 0 0,       % Red border around internal links
    citebordercolor=0 1 0,       % Green border around citations
    urlbordercolor=0 1 1,        % Cyan border around URLs
    filebordercolor=0 .5 .5,     % Teal border around file links
    pdfpagemode=UseOutlines,     % Open the PDF with bookmarks visible
}

\newcommand{\cooltext}[1]{%
  \texttt{\colorbox{gray!15}{\textcolor{black}{#1}}}%
}

% Configure bullet points with customized size
\setlist[itemize,1]{label={\Large\textbullet}} % First level: larger bullet point
\setlist[itemize,2]{label={\Large\textbullet}, leftmargin=2em} % Second level: larger bullet point and indented by 2em

% Remove ACM-specific references and permissions
\setcopyright{none} % Disable copyright
\makeatletter
\@printpermissionfalse
\@printcopyrightfalse
\@acmownedfalse
\makeatother

% Remove ACM reference format
\settopmatter{printacmref=false}
\renewcommand\footnotetextcopyrightpermission[1]{}

% Optional: Plain page style
\pagestyle{plain}

% TODO: check
% \BibTeX command to typeset BibTeX logo in the docs
\AtBeginDocument{%
  \providecommand\BibTeX{{%
    Bib\TeX}}}

% TODO: check
% Define paths for graphics and bibliography
% \graphicspath{{template/column-format-template/}}
% \bibliographystyle{template/column-format-template/ACM-Reference-Format}

% TODO: check
% \graphicspath{{figures/}}                 % Single directory
% \graphicspath{{fig1/}{fig2/}}             % Multiple directories

% TODO: check
% \bibliographystyle{ACM-Reference-Format}  % Standard ACM format
% \bibliographystyle{plain}                 % Basic format
% \bibliographystyle{ieeetr}                % IEEE format

% Optional: Customize section titles
% \titleformat{\section}
%   {\LARGE}                      % Large, not bold
%   {\thesection}                 % Section number
%   {0.5em}                       % Space between number and title
%   {}                            % Code before title body

% Optional: Customize subsection titles
% \titleformat{\subsection}
%   {\Large}                      % Large (but smaller than section), not bold
%   {\thesubsection}              % Subsection number
%   {0.5em}                       % Space between number and title
%   {}                            % Code before title body

% Optional: Adjust spacing before and after sections
% \titlespacing*{\section}{0pt}{2.5ex plus 1ex minus .2ex}{1.5ex plus .2ex}
% \titlespacing*{\subsection}{0pt}{2.25ex plus 1ex minus .2ex}{1.5ex plus .2ex}

\begin{document}

% Title
\title{Laboratory 2 Report}

% Define authors and their affiliations

\author{Andrea Botticella}
\authornote{The authors collaborated closely in developing this project.}
\email{andrea.botticella@studenti.polito.it}
\affiliation{%
  \institution{Politecnico di Torino}
  \city{Turin}
  \country{Italy}
}

\author{Elia Innocenti}
\authornotemark[1]
\email{elia.innocenti@studenti.polito.it}
\affiliation{%
  \institution{Politecnico di Torino}
  \city{Turin}
  \country{Italy}
}

\author{Simone Romano}
\authornotemark[1]
\email{simone.romano2@studenti.polito.it}
\affiliation{%
  \institution{Politecnico di Torino}
  \city{Turin}
  \country{Italy}
}

% Short author list for page headers
\renewcommand{\shortauthors}{Botticella, Innocenti, and Romano}

% CCS Concepts
% FIXME: update and check FOR EACH REPORT
\begin{CCSXML}
<ccs2012>
   <concept>
       <concept_id>10010147.10010257.10010321.10010337</concept_id>
       <concept_desc>Computing methodologies~Supervised learning by classification</concept_desc>
       <concept_significance>500</concept_significance>
   </concept>
   <concept>
       <concept_id>10010147.10010257.10010321.10010335</concept_id>
       <concept_desc>Computing methodologies~Unsupervised learning</concept_desc>
       <concept_significance>500</concept_significance>
   </concept>
   <concept>
       <concept_id>10010147.10010257.10010339</concept_id>
       <concept_desc>Computing methodologies~Natural language processing</concept_desc>
       <concept_significance>400</concept_significance>
   </concept>
   <concept>
       <concept_id>10010147.10010257.10010236</concept_id>
       <concept_desc>Computing methodologies~Machine learning</concept_desc>
       <concept_significance>400</concept_significance>
   </concept>
   <concept>
       <concept_id>10010147.10010528.10010531</concept_id>
       <concept_desc>Computing methodologies~Machine learning approaches</concept_desc>
       <concept_significance>300</concept_significance>
   </concept>
   <concept>
       <concept_id>10002978.10003006.10011634</concept_id>
       <concept_desc>Security and privacy~Intrusion detection systems</concept_desc>
       <concept_significance>200</concept_significance>
   </concept>
</ccs2012>
\end{CCSXML}

% FIXME: update and check FOR EACH REPORT
\ccsdesc[500]{Computing methodologies~Supervised learning by classification}
\ccsdesc[500]{Computing methodologies~Unsupervised learning}
\ccsdesc[400]{Computing methodologies~Natural language processing}
\ccsdesc[400]{Computing methodologies~Machine learning}
\ccsdesc[300]{Computing methodologies~Machine learning approaches}
\ccsdesc[200]{Security and privacy~Intrusion detection systems}

% Keywords
% FIXME: update and check FOR EACH REPORT
\keywords{Machine learning, supervised learning, unsupervised learning, language models, text classification, clustering, intent classification, SSH shell attacks, security log analysis}

% abstract.tex

\begin{abstract}

    This is the abstract of the report.
    
\end{abstract}


\maketitle

% Table of Contents
% TODO: check
\setcounter{tocdepth}{1}
\tableofcontents

% Include the sections you have in your sections directory
% introduction.tex

% Section Title
\section{INTRODUCTION} \label{sec:introduction}

    % text

% conclusion.tex

% Section Title
\section{CONCLUSIONS} \label{sec:conclusions}

    % text


% Include the appendix
\clearpage
\appendix
% appendix.tex

% Section Title
\section{APPENDIX} \label{sec:appendix}

    % text


% Include the bibliography file
% TODO: check
\nocite{*} % Include all references from the .bib file
\bibliographystyle{template/ACM-Reference-Format}
\bibliography{bibliography/references}

\end{document}
